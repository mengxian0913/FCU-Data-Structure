% !TeX encoding = UTF-8
% !TeX program = xelatex
\documentclass[12pt, a4paper]{article}
\usepackage[CheckSingle, CJKmath]{xeCJK}
\usepackage{fontspec}
\usepackage{graphicx} % 插入圖片
\usepackage{caption}
\usepackage{enumerate}
\usepackage{setspace}
\usepackage{array} % 製作表格必須的宏包
\usepackage{tabularx} % 自動調整列寬的表格宏包
\usepackage{adjustbox}
\usepackage{geometry} 

\setCJKfamilyfont{heiti}{Heiti TC}
\CJKfamily{heiti}
\setmainfont{Arial}
\setstretch{1.5}


\usepackage{amsthm}                             %定義,例題
\usepackage{amssymb}
\usepackage{fancyhdr}                           %設定頁首頁尾
\usepackage{titlesec}
\usepackage{amsmath}
\usepackage{CJKulem}
\usepackage{amsmath, courier, listings, fancyhdr, graphicx}
\usepackage{listings}
\usepackage[utf8]{inputenc}
\usepackage[english]{babel}
\usepackage{times}
\usepackage{listings}
\usepackage{color}
\usepackage[x11names]{xcolor}

% \headsep=2pt
% \textheight=740pt
% \footskip=0pt
% \voffset=-50pt
% \textwidth=545pt
% \marginparsep=0pt
% \marginparwidth=0pt
% \marginparpush=0pt
% \oddsidemargin=0pt
% \evensidemargin=0pt
% \hoffset=-42pt


\setmainfont[
    AutoFakeSlant,
    BoldItalicFeatures={FakeSlant},
    UprightFont={*},
    BoldFont={*-Bold}
]{Inconsolata}

\setmonofont[
    AutoFakeSlant,
    BoldItalicFeatures={FakeSlant},
    BoldFont={*-Bold}
]{Inconsolata}


\makeatletter
\lst@CCPutMacro\lst@ProcessOther {"2D}{\lst@ttfamily{-{}}{-{}}}
\@empty\z@\@empty
\makeatother
\lstset{                                        % Code顯示
    language=C++,                               % the language of the code
    basicstyle=\footnotesize\ttfamily,          % the size of the fonts that are used for the code
    numbers=left,                               % where to put the line-numbers
    numberstyle=\scriptsize,                    % the size of the fonts that are used for the line-numbers
    stepnumber=1,                               % the step between two line-numbers. If it's 1, each line  will be numbered
    numbersep=5pt,                              % how far the line-numbers are from the code
    backgroundcolor=\color{white},              % choose the background color. You must add \usepackage{color}
    showspaces=false,                           % show spaces adding particular underscores
    showstringspaces=false,                     % underline spaces within strings
    showtabs=false,                             % show tabs within strings adding particular underscores
    frame=false,                                % adds a frame around the code
    tabsize=2,                                  % sets default tabsize to 2 spaces
    captionpos=b,                               % sets the caption-position to bottom
    breaklines=true,                            % sets automatic line breaking
    breakatwhitespace=true,                     % sets if automatic breaks should only happen at whitespace
    escapeinside={\%*}{*)},                     % if you want to add a comment within your code
    morekeywords={*},                           % if you want to add more keywords to the set
    keywordstyle=\bfseries\color{Blue1},
    commentstyle=\itshape\color{Red1},
    stringstyle=\itshape\color{Green4},
}






\begin{document}
  \begin{center}
    {\Huge 資料結構實習} \\[2.5cm]
    {\Huge 11/03 作業報告} \\[1.5cm]
    {\Huge Stack 實作 (e, s) segment} \\ [4.5cm]
    \hspace{.6in}
    \begin{minipage}[t]{.4\linewidth}
      {\Large 班級:資訊二甲}\\[0.5cm]
      {\Large 學號:D1109023}\\[0.5cm]
      {\Large 姓名:楊孟憲}
    \end{minipage}    
  \end{center}

  \newpage

  \begin{samepage}
    \fontsize{16pt}{18pt} \selectfont  
    % 生成目錄
    \tableofcontents
    \normalfont
  \end{samepage}
  
  \newpage


  \section{\fontsize{20pt}{22pt}\selectfont 引言}
  \begin{samepage}
    \fontsize{16pt}{18pt} \selectfont
      Stack 是一個只有一個開口的單向輸入輸出的資料結構,在日常生活中可以使用該資料結構維護許多演算法。
      插入以及拿取都只需要 $O(1)$ 的時間。 以下是利用 LinkedList 實作 Stack 的方法。(以下是使用 C 實作 Stack 的 push/pop)
      \lstinputlisting[language=C++]{../code/Stack.c}
  \end{samepage}

  % 第一個章節

  \section{\fontsize{20pt}{22pt}\selectfont 題目敘述}
  \begin{samepage}
    \fontsize{16pt}{18pt} \selectfont
        \textbf{題意說明:}如果在一個字串當中其頭一個字母為 E,最後一個字母為 S,而兩個字母 中間不包含任何 E 或 S
        字母的話,則稱為 ES 字串。
        請利用堆疊 (Stacks) 的原理,撰寫出一個程式,從檔案(input.txt)讀入一段文章或是字串,然後消去所有
        可能的 ES 字串,使得消去後的字串輸出 不包含任何 ES 字串。
  \end{samepage}
  % 第二個章節

  \section{\fontsize{20pt}{22pt}\selectfont 作法}
  % 第二章節的子章節
  \begin{samepage}
    \fontsize{16pt}{18pt} \selectfont
      使用 C++ 的 STL Stack 實作。
      遍歷字串的每一個字元,當前字元為 's' 時,判斷 stack 是否為空並且如果前面有 'e' 的話 (cnt > 0),就不斷地把先前的字元從 stack 拿出來,直到 now == 'e',並且將 cnt-1,這能夠確保之後的 e, s 區間能被正確的移除。
      否則,就把 's' 放進 stack 裡。 \\
      如果當前字元不為 's' 的話,就放進 stack 裡,並且判斷如果當前字元為 'e' 的話,就 cnt++;\\
      \textbf{範例程式碼}  
    \lstinputlisting[language=C++]{../code/a.cpp}
    \normalfont
  \end{samepage}

  \section{\fontsize{20pt}{22pt}\selectfont 執行結果}
  % 第三個章節
      \fontsize{16pt}{18pt} \selectfont
        輸入輸出結果:
        \begin{enumerate}
          \item 輸入:
            \lstinputlisting{../code/input.txt}
          \item 輸出:
            \lstinputlisting{../code/output.txt}
        \end{enumerate}
      \normalsize

  \section{\fontsize{20pt}{22pt}\selectfont 心得與討論}
  \begin{samepage}
    \fontsize{16pt}{18pt} \selectfont
      這次的作業利用 Stack 的原理來完成,是一個常見的模板題,實作起來沒有什麼問題,這讓我更加了解 Stack 的運作原理。
    \normalfont
  \end{samepage}
  % 最後一個章節
\end{document}
